\documentclass{article}
\usepackage[french]{babel}
\addto\captionsfrench{
  \renewcommand{\contentsname}{Table of Contents}
  \renewcommand{\refname}{References}
}

\usepackage[utf8]{inputenc}
\usepackage[T1]{fontenc}
\usepackage{geometry}
\geometry{margin=2cm}

\usepackage{xcolor}
\usepackage{url}
\usepackage[colorlinks]{hyperref}
\hypersetup{citecolor=red!90, urlcolor=blue!80, linkcolor=blue!80}

\usepackage{amsmath, amsfonts, amssymb}
\usepackage{relsize}

\setlength{\parindent}{2em}
\setlength{\parskip}{5pt}
\renewcommand{\baselinestretch}{1.2}

\usepackage{makeidx}
\makeindex

\newcommand{\B}{\mathrm{B}}
\newcommand{\HAU}{\mathrm{H}}
\newcommand{\SL}{\mathrm{S}}
\newcommand{\U}{\mathrm{U}}
\newcommand{\V}{\mathrm{V}}
\newcommand{\M}{\mathrm{M}}
\newcommand{\I}{\mathrm{I}}
\newcommand{\J}{\mathrm{J}}
\newcommand{\Q}{\mathrm{Q}}

\newcommand{\HD}{\mathrm{HD}}
\newcommand{\RB}{\mathrm{RB}}
\newcommand{\EF}{\mathrm{EF}}
\newcommand{\DF}{\mathrm{DF}}

\newcommand{\QS}{\mathrm{QS}}
\newcommand{\QSX}{\mathrm{QSX}}
\newcommand{\QSY}{\mathrm{QSY}}
\newcommand{\QSBL}{\mathrm{QSBL}}
\newcommand{\QSBLX}{\mathrm{QSBLX}}
\newcommand{\QSBLY}{\mathrm{QSBLY}}
\newcommand{\QSSUSP}{\mathrm{QSSUSP}}
\newcommand{\QSSUSPX}{\mathrm{QSSUSPX}}
\newcommand{\QSSUSPY}{\mathrm{QSSUSPY}}

\newcommand{\US}{\mathrm{US}}
\newcommand{\DMAX}{\mathrm{DMAX}}
\newcommand{\TAU}{\mathrm{TAU}}
\newcommand{\IF}{\mathrm{if}\;}

\usepackage{hyperref}
\usepackage{multicol}

\title{Equations integrated in \textsc{PyTelTools}}
\author{Yishu \textsc{Wang}, Luc \textsc{Duron}}
\date{August 2017}

\begin{document}
\shorthandoff{:}

\maketitle
\small
\tableofcontents\newpage
\printindex
\normalsize

\section{Variables notations}

Notation of 2D and 3D variables are described on this wiki page : \url{https://github.com/CNR-Engineering/PyTelTools/wiki/Notations-of-variables}.
Variable notations used in this document are based on the \texttt{varID} column of the tables presented on latter webpage.

\section{Constants}

Equations may contain following constants:

\begin{center}
  \begin{tabular}{ | c | c | c | }
    \hline
    Notation  & Description & Value and unit \\
    \hline
    $g$      & gravitational accelaration on earth & $9.80665 \; m.s^{-2}$ \\
    $\rho$   & water density                       & $1000 \; kg.m^{-3}$ \\
    $\kappa$ & Von Kármán constant                 & 0.4 \\
    \hline
  \end{tabular}
\end{center}

These constants are set in \texttt{slf/variable/variables\_utils.py}.

\section{Equations in 2D}

\begin{multicols}{2}

\subsection{Hydraulic variables}
\[\begin{aligned}
& \HAU = \SL - \B\\
& \SL = \HAU + \B\\
& \B = \SL - \HAU\\
& \M = \sqrt{\U^2+\V^2}\\
& \I = \HAU \times \U\\
& \J = \HAU \times \V\\
& \mathrm{Q} = \sqrt{\I^2 + \J^2}\\
& \mathrm{C} = \sqrt{g\,\HAU}\\
& \mathrm{F} = \frac{\M}{\mathrm{C}}
\end{aligned}\]

\subsection{Sediment transport variables}
\[\begin{aligned}
& \HD = \B - \RB\\
& \B = \HD - \RB\\
& \RB = \B - \HD\\
& \QS = \sqrt{\QSX ^2 + \QSY ^2}\\
& \QS = \EF + \DF\\
& \QSBL = \sqrt{\QSBLX ^2 + \QSBLY ^2}\\
& \QSSUSP = \sqrt{\QSSUSPX ^2 + \QSSUSPY ^2}
\end{aligned}\]

\end{multicols}

\subsection{Friction velocity}
\[\US = \sqrt{\frac{1}{2}\,C_f\,\M^2}\]

\begin{center}
  \begin{tabular}{ | c | c | }
    \hline
    Law       & Coefficient \\
    \hline
    Chézy     & $C$ \\
    Strickler & $K$ \\
    Manning   & $m$ \\
    Nikuradse & $k_S$ \\
    \hline
  \end{tabular}
\end{center}

\begin{multicols}{2}
\subsubsection{Chézy}
\[C_f = \frac{2 g}{C^2}\]

\subsubsection{Strickler}
\[ C_f = \frac{2 g}{K^2 \HAU^{1/3}}\]

\subsubsection{Manning}
\[ C_f = \frac{2 g m^2}{\HAU^{1/3}}\]

\subsubsection{Nikuradse}

\[ C_f = \frac{2 \kappa^2}{ \left[ \log\left(\frac{30}{e^1} \frac{\HAU}{k_S}\right) \right] ^2} \]
\end{multicols}


\subsection{Bed shear stress, Rouse number and diameter}
\[\begin{aligned}
& \TAU = \tau = \rho \,\US^2\\
& R_0 = \frac{w_s}{\kappa\, \US}\\
& \mathrm{FROTP} = \mathrm{MU} \; \DMAX\\
& \DMAX = \begin{cases}
     1.4593 \times \tau^{0.979} &\IF   \tau > 3.4 \\
     1.2912 \times \tau^2 + 1.3572 \tau - 0.1154
    &\IF 0.1 < \tau \leqslant 0.34 \\
     0.9055 \times \tau^{1.3178} &\IF \tau \leqslant 0.1
  \end{cases}
\end{aligned}\]


\section{Equations in 3D}

\[\begin{aligned}
& \M = \sqrt{\U^2+\V^2+\mathrm{W}^2}\\
& \mathrm{NU} = \sqrt{\mathrm{NUX}^2+\mathrm{NUY}^2+\mathrm{NUZ}^2}
\end{aligned}\]

\end{document}
