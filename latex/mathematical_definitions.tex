\documentclass{article}
\usepackage[french]{babel}
\addto\captionsfrench{
  \renewcommand{\contentsname}{Table of Contents}
  \renewcommand{\refname}{References}
}

\usepackage[utf8]{inputenc}
\usepackage[T1]{fontenc}
\usepackage{geometry}
\geometry{hmargin=2.7cm, vmargin=3cm}
\usepackage{indentfirst}


\usepackage{xcolor}
\usepackage{url}
\usepackage[colorlinks]{hyperref}
\hypersetup{citecolor=red!90, urlcolor=blue!80, linkcolor=blue!80}

\usepackage{amsmath, amssymb, amsfonts, amsthm}
\usepackage{relsize}

\setlength{\parindent}{2em}
\setlength{\parskip}{5pt}
\renewcommand{\baselinestretch}{1.2}

\usepackage{url}
\usepackage{enumerate}
\usepackage{graphicx}
% \usepackage[subpreambles=true]{standalone}
\usepackage{standalone}

\theoremstyle{definition}
\newtheorem{defn}{Definition}
\newtheorem{prop}{Proposition}
\newtheorem{remark}{Remark}

\usepackage{tikz}

\usepackage{makeidx}
\makeindex

\newcommand{\dd}{\,\mathrm{d}}
\newcommand{\RR}{\mathbb{R}}
\newcommand{\EE}{\mathbb{E}}
\newcommand{\MM}{\mathcal{M}}
\newcommand{\VV}{\mathcal{V}}
\newcommand{\TT}{\mathcal{T}}
\newcommand{\area}{\mathrm{area}\hspace{1pt}}
\newcommand{\size}{\mathrm{size}\hspace{1pt}}
\newcommand{\flux}{\mathrm{flux\hspace{1pt}}}
\newcommand{\vol}{\mathrm{volume\hspace{1pt}}}
\newcommand{\ST}{\mathrm{ST}}
\newcommand{\card}{\mathrm{card\hspace{1pt}}}
\newcommand{\absvol}{\mathrm{volume\hspace{1pt}}^\mathrm{abs}}
\newcommand{\quadvol}{\mathrm{volume\hspace{1pt}}^\mathrm{quad}}

\title{Mathematical definitions for \textsc{PyTelTools}}
\author{Yishu \textsc{Wang}, Luc \textsc{Duron}}
\date{August 2017}

\begin{document}
\shorthandoff{:}

\maketitle
\small
\tableofcontents\newpage
\printindex
\normalsize

\section{Introduction}
\index{Basic definitions}
We call informally a \textbf{2D mesh} $\MM$, or simple \textbf{mesh}, a finite network where
\begin{itemize}
\item an \textbf{element} $\TT$ is a triangle in $\RR^2$; we write $\TT\in\MM$; its vertices $P_{1,2,3}$ are called \textbf{points} of the mesh;\index{Basic definitions!Element}\index{Basic definitions!Point}
\item a \textbf{connection} is an common edge (two common vertices) between two elements;
\end{itemize}
such that
\begin{itemize}
\item every two elements can either intersect at one common vertex, intersect at one common edge (two common vertices), or have empty intersection;
\item every element has at least one common edge with another element (the mesh is \textbf{connected}).
\end{itemize}
\index{Basic definitions!Mesh}
Furthermore, for some integer $t$ between 0 and $T_{\max}$,
a mesh can have a \textbf{valuation} $\VV_t$ at time $t$, which assigns to each point of the mesh a real number, called the \textbf{value} of the point.
\index{Basic definitions!Valuation, Value of a point}

Other useful notations include
\begin{itemize}
\item $\area(\TT)$ is the area of the triangle $\TT$;
\item $\area(\MM)=\sum_{\TT\in\MM}\area(\TT)$ is the total area of elements in the mesh $\MM$;
\item $\size(\MM)$ is the number of elements in the mesh $\MM$.
\end{itemize}

\section{Interpolation}
\index{Interpolation}
\begin{defn}
Let $\TT$ be a triangle in $\RR^2$ of vertices $P_{1,2,3}$, for a point $Q\in\TT$ (in the interior or on the boundary), the \textbf{barycentric coordinates} of $Q$ is equal to
\[\lambda_1 = \frac{\area(\TT_1)}{\area(\TT)},\quad\lambda_2 = \frac{\area(\TT_2)}{\area(\TT)},\quad\lambda_3 = \frac{\area(\TT_3)}{\area(\TT)}\]
where $\TT_1,\,\TT_2,\,\TT_3$ are triangles in $\RR^2$ defined by the vertices $P_2\,P_3\,Q$, $P_1\,P_3\,Q$ and $P_1\,P_2\,Q$
(when $Q$ is on the boundary of $\TT$, one or two of the triangles $\TT_1,\,\TT_2,\,\TT_3$ are considered to be empty triangle).
\end{defn}
\index{Interpolation!Barycentric coordinates}

\begin{defn}
Given a mesh $\MM$ valuated at $\VV_t$ and an element $\TT\in\MM$ of vertices $P_{1,2,3}$, the \textbf{interpolated value of a point} $Q\in\TT$ is equal to the dot product of the barycentric coordinates of $Q$ and the values of $P_{1,2,3}$:
\[\lambda_1\,\VV_t(P_1) + \lambda_2\,\VV_t(P_2) + \lambda_3\,\VV_t(P_3)\]
\end{defn}
\index{Interpolation!Interpolated value of a point}


\section{Sub-triangulation}
\subsection{Definition}
\begin{defn}
A \textbf{sub-triangulation} $(\MM^*,\,\VV_t^*)$ of a mesh $\MM$ valuated at $\VV_t$ is the result of the operation that, for every element $\TT\in\MM$ of vertices $P_{1,2,3}$,
\begin{itemize}
\item adds a point $Q$ in the interior of $\TT$ and splits $\TT$ into three elements $\TT_1,\,\TT_2,\,\TT_3$ of vertices $P_2\,P_3\,Q$, $P_1\,P_3\,Q$ and $P_1\,P_2\,Q$, called the splits of $\TT$ at $Q$;
\item extends $\VV_t$ to $Q$ and assigns to $Q$ the interpolated value at $Q$.
\end{itemize}
We note $\ST(\MM,\,\VV_t)$ the set of all sub-triangulations of $\MM$ valuated at $\VV_t$.
\end{defn}
\index{Sub-triangulation!Sub-triangulation}

\subsection{Invariance under sub-triangulation}
\begin{defn}
Given a mesh $\MM$ valuated at $\VV_t$, a function $f$ of $(\MM,\,\VV_t)$ is \textbf{invariant under sub-triangulation} if $f$ is constant in the set of all sub-triangulations of $\MM$ valuated at $\VV_t$:
\[\forall (\MM^*,\,\VV_t^*)\in\ST(\MM,\,\VV_t),\quad f(\MM^*,\,\VV_t^*) = f(\MM,\,\VV_t).\]
\end{defn}
\index{Sub-triangulation!Invariance under sub-triangulation}


\section{Flux}
\index{Flux}

TODO: coming soon

Define mass flux

Define simplified fluxes :
\begin{itemize}
  \item line flux
  \item area flux
\end{itemize}


\section{Volume}
\index{Volume}

\subsection{Volume of an element}
\begin{defn}
Given a mesh $\MM$ valuated at $\VV_t$, the \textbf{volume} $\vol_t(\TT,\,\VV_t)$ \textbf{of an element} $\TT\in\MM$ (also called \textbf{net volume}) of vertices $P_{1,2,3}$ is the product of the area of $\TT$ by one third the sum of the values of its three vertices:
\[\vol_t(\TT,\,\VV_t) = \frac{\area(\TT)}{3}
\sum_{P_i} \VV_t(P_i). \]
\label{def:volumeT}\end{defn}
\index{Volume!Volume of an element}


\subsection{Explanation}
The definition of the volume of an element $\TT$ valuated at $\VV_t$ can be easily understood by picturing the \textbf{truncated right triangular prism} $\overline{\TT}_t$ in $\RR^3$ naturally induced by the valuation $\VV_t$.
\index{Volume!Truncated right triangular prism}

\begin{figure}[ht!]
\centering \includestandalone[mode=buildnew]{prism}
  \caption{The induced prism $\overline{\TT}_t$ of an element $\TT$ valuated at $\VV_t$}
  \label{fig:prism}
\end{figure}

If we note $(x_1,\,y_1),\,(x_2,\,y_2),\,(x_3,\,y_3)$ the coordinates of the three vertices $P_{1,2,3}$ of $\TT$ in a Cartesian coordinate system, then the prism $\overline{\TT}_t$ is defined by the following six vertices (figure \ref{fig:prism}):
\begin{itemize}
\item the \textbf{projected points} $\overline{P}_{1,2,3}$ of coordinates $(x_1,\,y_1,\,0),\,(x_2,\,y_2,\,0),\,(x_3,\,y_3,\,0)$ which form the \textbf{base triangle} of the prism,
\item the \textbf{valuated points} $\overline{P}_{1,2,3}^{\VV_t}$ of coordinates $(x_1,\,y_1,\,\VV_t(P_1)),\,(x_2,\,y_2,\,\VV_t(P_2)),\,(x_3,\,y_3,\,\VV_t(P_3))$ which form the \textbf{upper triangle} of the prism.
\end{itemize}
\index{Volume!Projected points, Valuated points}
\index{Volume!Base triangle, Upper triangle}

This prism is truncated because the upper triangle is not necessarily parallel to the base triangle; it is also a right prism because the three lateral edges is always perpendicular to the base triangle.

It is known from the classical geometry that the volume of a truncated right triangular prism is equal to the product of its base by one third the sum of its lateral edges. The volume of an element $\TT$ valuated at $\VV_t$ in definition \ref{def:volumeT} corresponds thus to the volume of its induced prism.

Since the valuation $\VV_t$ can take positive and negative values, the upper triangle does not always lie on the half-space $z >0$ or $z<0$. When the upper triangle intersects the base triangle, the induced prism $\overline{\TT}_t$ \textbf{is not a polyhedron}, but the volume defined above is still valid.

\subsection{Volume of a mesh}
\begin{defn}
The \textbf{volume} $\absvol_t(\MM,\,\VV_t)$ \textbf{of a mesh} $\MM$ valuated at $\VV_t$ is the sum of the volumes of its elements:
\[\vol_t(\MM,\,\VV_t) = \sum_{\TT\in\MM}\vol_t(\TT,\,\VV_t).\]
\label{def:VolumeM}\end{defn}
\index{Volume!Volume of a mesh}

\begin{prop}
The volume of a mesh $\MM$ valuated at $\VV_t$ defined in definition \ref{def:VolumeM} is invariant under sub-triangulation.\label{prop:volume-invariant}
\end{prop}

\begin{proof}[Proof]
Let $\MM$ be a mesh valuated at $\VV_t$ and $(\MM^*,\,\VV_t^*) \in \ST(\MM,\,\VV_t)$. Let $\TT$ be any element of $\MM$ and $\TT_1,\,\TT_2,\,\TT_3$ its splits in $\MM^*$ at the point $Q$.
First we write down the total volume of the three splits:
\begin{align*}
& \sum_{k=1}^3\vol_t(\TT_k,\,\VV_t^*)\\ & =
   \frac{\area(\TT_1)}{3}\left(\VV_t^*(P_2)+ \VV_t^*(P_3) + \VV_t^*(Q)
\right)
 +  \frac{\area(\TT_2)}{3}\left(\VV_t^*(P_1) + \VV_t^*(P_3)+ \VV_t^*(Q)
\right)\\ &
\hspace{8pt} +  \frac{\area(\TT_3)}{3}\left(\VV_t^*(P_1)+ \VV_t^*(P_2) + \VV_t^*(Q)
\right) \\ & =
\frac{1}{3}\,\mathlarger{\mathlarger{\sum}}_{k=1}^3\left(\VV_t^*(P_k)\sum_{\substack{j=1,2,3 \\ j\neq k}}\area(\TT_j)\right) + \frac{1}{3}\,\VV_t^*(Q)\,\sum_{k=1}^3\area(\TT_k).
\end{align*}
By the definition of sub-triangulation, the values $\VV_t^*(P_k) = \VV_t(P_k)$ and the value of $Q$ is equal to \[\VV_t^*(Q) = \sum_{k=1}^3\lambda_k\,\VV_t(P_k) = \frac{1}{\area(\TT)}\sum_{k=1}^3\area(\TT_k)\,\VV_t(P_k)
=  \frac{\sum_{k=1}^3\area(\TT_k)\,\VV_t(P_k)}{\sum_{k=1}^3\area(\TT_k)},\]
thus,
\begin{align*}\sum_{k=1}^3\vol_t(\TT_k,\,\VV_t^*) & =
\frac{1}{3}\,\mathlarger{\mathlarger{\sum}}_{k=1}^3\left(\VV_t(P_k)\sum_{\substack{j=1,2,3 \\ j\neq k}}\area(\TT_j)\right)
+ \frac{1}{3}\,\sum_{k=1}^3\area(\TT_k)\,\VV_t(P_k)
\\ & = \frac{1}{3}\,\mathlarger{\mathlarger{\sum}}_{k=1}^3\left(\VV_t(P_k)\sum_{j=1}^3\area(\TT_j)\right)
= \frac{\area(\TT)}{3} \sum_{k=1}^3 \VV_t(P_i) \\ & = \vol_t(\TT,\,\VV_t) .
\end{align*}
Therefore,
\[\vol_t(\MM^*,\,\VV_t^*) = \sum_{\TT\in\MM}\left(\sum_{\substack{\TT_k\\ \mathrm{splits\;of\;} \TT}} \vol_t(\TT_k,\,\VV_t^*)\right) = \sum_{\TT\in\MM} \vol_t(\TT,\,\VV_t) = \vol_t(\MM,\,\VV_t).\]
\end{proof}


\subsection{Positive and negative volumes of an element}
\index{Volume!Positive and negative volumes of an element}
\begin{defn}
Given a mesh $\MM$ valuated at $\VV_t$, the \textbf{positive volume} $\vol_t^+(\TT,\,\VV_t)$ \textbf{of an element} $\TT\in\MM$ of vertices $P_{1,2,3}$, is equal to
\begin{itemize}
\item 0, if $\VV_t(P_i) \leqslant 0$ for $i=1,2,3$ (the upper triangle lies in the half-space $z\leqslant0$);
\item its net volume $\vol_t(\TT,\,\VV_t)$, if $\VV_t(P_i) \geqslant 0$ for $i=1,2,3$ (the upper triangle lies in the half-space $z\geqslant 0$);
\item otherwise, the upper triangle intersect the plane $z=0$ at a segment $Q_1\,Q_2$ (figure \ref{fig:positive-volume}). There are two cases:
\begin{itemize}
\item if $\TT$ has only one vertex, w.l.o.g. $P_1$, such that $\VV_t(P_1) > 0$, the positive volume is equal to the volume of the tetrahedron defined by $Q_1,\,Q_2$, the projected point $\overline{P}_1$ and the valuated point $\overline{P}_1^{\VV_t}$ of $P_1$;
\item otherwise, $\TT$ has only one vertex, w.l.o.g. $P_1$, such that $\VV_t(P_1) < 0$. The positive volume is equal to the difference between the net volume $\vol_t(\TT,\,\VV_t)$ and the volume of the tetrahedron defined by $Q_1,\,Q_2$, the projected point $\overline{P}_1$ and the valuated point $\overline{P}_1^{\VV_t}$ of $P_1$.
\end{itemize}
\end{itemize}
\end{defn}
\begin{figure}[ht!]
\centering \includestandalone[mode=buildnew]{positive-volume}
  \caption{When the upper triangle intersects the base triangle at a segment $Q_1\,Q_2$}
  \label{fig:positive-volume}
\end{figure}

\begin{defn}
Given a mesh $\MM$ valuated at $\VV_t$, the \textbf{negative volume} $\vol_t^-(\TT,\,\VV_t)$ \textbf{of an element} $\TT\in\MM$ is equal to the difference between its net volume and its positive volume:
\[\vol_t^-(\TT,\,\VV_t) =  \vol_t(\TT,\,\VV_t) -  \vol_t^+(\TT,\,\VV_t).\]
\end{defn}

\subsection{Absolute and quadratic volumes of an element, of a mesh}
\begin{defn}
Given a mesh $\MM$ valuated at $\VV_t$, the \textbf{absolute volume} $\absvol_t(\TT,\,\VV_t)$ \textbf{of an element} $\TT\in\MM$ of vertices $P_{1,2,3}$ is the product of the area of $\TT$ by one third the sum of the absolute values of its three vertices:
\[\absvol_t(\TT,\,\VV_t) = \frac{\area(\TT)}{3}
\sum_{k=1}^3 \left| \VV_t(P_i) \right|. \]
\label{def:absVolumeT}\end{defn}
\index{Volume!Absolute volume of an element}

\begin{defn}
The \textbf{absolute volume} $\absvol_t(\MM,\,\VV_t)$ \textbf{of a mesh} $\MM$ valuated at $\VV_t$ is the sum of the absolute volumes of its elements:
\[\absvol_t(\MM,\,\VV_t) = \sum_{\TT\in\MM}\absvol_t(\TT,\,\VV_t).\]
\label{def:absVolumeM}\end{defn}
\index{Volume!Absolute volume of a mesh}

\begin{prop}
The absolute volume of a mesh $\MM$ valuated at $\VV_t$ defined in definition \ref{def:absVolumeM} is invariant under sub-triangulation.\label{prop:absvolume-invariant}
\end{prop}
\begin{proof}[Proof]
Analogous to the proof of proposition \ref{prop:volume-invariant}.
\end{proof}


\begin{defn}
Given a mesh $\MM$ valuated at $\VV_t$, the \textbf{quadratic volume} $\quadvol_t(\TT,\,\VV_t)$ \textbf{of an element} $\TT\in\MM$ is equal to the volume of $\TT$ valuated at $\VV_t^2$:
\[\quadvol_t(\TT,\,\VV_t) =\vol_t(\TT,\,\VV_t^2) =  \frac{\area(\TT)}{3}
\sum_{k=1}^3  \VV_t^2(P_i). \]
\label{def:quadVolumeT}\end{defn}
\index{Volume!Quadratic volume of an element}

\begin{defn}
The \textbf{quadratic volume} $\absvol_t(\MM,\,\VV_t)$ \textbf{of a mesh} $\MM$ valuated at $\VV_t$ is the square root of the sum of the quadratic volumes of its elements:
\[\quadvol_t(\MM,\,\VV_t) = \sqrt{\sum_{\TT\in\MM}\quadvol_t(\TT,\,\VV_t)}.\]
\label{def:quadVolumeM}\end{defn}
\index{Volume!Quadratic volume of a mesh}

\begin{prop}
The quadratic volume of a mesh $\MM$ valuated at $\VV_t$ defined in definition \ref{def:quadVolumeM} is invariant under sub-triangulation.\label{prop:quadvolume-invariant}
\end{prop}
\begin{proof}[Proof]By definition,
\begin{align*}
\quadvol_t(\MM,\,\VV_t) & = \sqrt{\sum_{\TT\in\MM}\quadvol_t(\TT,\,\VV_t)}\\
& = \sqrt{\sum_{\TT\in\MM}\vol_t(\TT,\,\VV_t^2)}\\
& = \sqrt{\vol_t(\MM,\,\VV_t^2)},
\end{align*}
yet by proposition \ref{prop:volume-invariant}, the volume of $\MM$ valuated at $\VV_t^2$ is invariant under sub-triangulation, thus the quadratic volume of $\MM$ valuated at $\VV_t$ is also invariant under sub-triangulation.
\end{proof}



\section{Comparison between two meshes}
\index{Comparison between two meshes}


\subsection{Comparison between two identical meshes}
\begin{defn}
Two meshes $\MM_1$ and $\MM_2$ are \textbf{identical} if the sets of their elements are equal:
\[\{\TT\in\MM_1\} = \{\TT\in\MM_2\}\].
\end{defn}
\index{Comparison between two meshes!Identical meshes}

\begin{defn}
A \textbf{reference mesh} is a mesh $\MM_0$ with a constant valuation $\VV_t = \VV_0$ for all time $t$.
\end{defn}
\index{Comparison between two meshes!Reference mesh}

\subsubsection{Mean signed deviation}
\begin{defn}
Let $\MM_0$ be a reference mesh with valuation $\VV_0$. For any mesh valuated at $\VV_t$ and identical to $\MM_0$, the \textbf{mean signed deviation (MSD)} with respect to $\MM_0$ is the function  $\mathrm{MSD}_{\MM_0,\VV_0}$ in $\RR_+$ that maps $(\MM_0,\,\VV_t)$ to the volume of $\MM_0$ valuated at $\VV_t-\VV_0$, divided by the total area of $\MM_0$:
\[\mathrm{MSD}_{\MM_0,\VV_0} (\MM_0,\,\VV_t) = \frac{\vol_t(\MM_0, \VV_t-\VV_0)}{\area(\MM_0)}.\]
\label{def:MSD}
\end{defn}
\index{Comparison between two meshes!Mean signed deviation}
\begin{prop}
The mean signed distance is invariant under sub-triangulation.
\end{prop}
\begin{proof}[Proof]
It is clear that a sub-triangulation does not change the total area of the mesh.
By proposition \ref{prop:volume-invariant}, the volume of $\MM_0$ valuated at $\VV_t-\VV_0$ is invariant under sub-triangulation. Therefore the mean signed deviation is also invariant under sub-triangulation.
\end{proof}


\subsubsection{Mean absolute deviation}
\begin{defn}
Let $\MM_0$ be a reference mesh with valuation $\VV_0$. For any mesh valuated at $\VV_t$ and identical to $\MM_0$, the \textbf{mean absolute error (MAD)} with respect to $\MM_0$ is the function  $\mathrm{MAD}_{\MM_0,\VV_0}$ in $\RR_+$ that maps $(\MM_0,\,\VV_t)$ to the absolute volume of $\MM_0$ valuated at $\VV_t-\VV_0$, divided by the total area of $\MM_0$:
\[\mathrm{MAD}_{\MM_0,\VV_0} (\MM_0,\,\VV_t) = \frac{\absvol_t(\MM_0, \VV_t-\VV_0)}{\area(\MM_0)}.\]
\end{defn}
\index{Comparison between two meshes!Mean absolute deviation}

\begin{prop}
The mean absolute distance is invariant under sub-triangulation.
\end{prop}
\begin{proof}[Proof]
It is clear that a sub-triangulation does not change the total area of the mesh.
By proposition \ref{prop:absvolume-invariant}, the absolute volume of $\MM_0$ valuated at $\VV_t-\VV_0$ is invariant under sub-triangulation. Therefore the mean absolute deviation is also invariant under sub-triangulation.
\end{proof}

\subsubsection{Root mean square deviation}
\begin{defn}
Let $\MM_0$ be a reference mesh with valuation $\VV_0$. For any mesh valuated at $\VV_t$ and identical to $\MM_0$, the \textbf{root mean square deviation (RMSD)} with respect to $\MM_0$ is the function  $\mathrm{RMSD}_{\MM_0,\VV_0}$ in $\RR_+$ that maps $(\MM_0,\,\VV_t)$ to the quadratic volume of $\MM_0$ valuated at $\VV_t-\VV_0$, divided by the total area of $\MM_0$:
\[\mathrm{RMSD}_{\MM_0,\VV_0} (\MM_0,\,\VV_t) = \frac{\quadvol_t(\MM_0, \VV_t-\VV_0)}{\area(\MM_0)}.\]
\end{defn}
\index{Comparison between two meshes!Root mean square deviation}

\begin{prop}
The root mean square deviation is invariant under sub-triangulation.
\end{prop}
\begin{proof}[Proof]
It is clear that a sub-triangulation does not change the total area of the mesh.
By proposition \ref{prop:quadvolume-invariant}, the quadratic volume of $\MM_0$ valuated at $\VV_t-\VV_0$ is invariant under sub-triangulation. Therefore the root mean square deviation is also invariant under sub-triangulation.
\end{proof}




\subsubsection{Element-wise signed deviation, deviation distribution}
Let $\MM_0$ be a reference mesh with valuation $\VV_0$. For any mesh valuated at $\VV_t$ and identical to $\MM_0$, the \textbf{deviation distribution function} $F_X$ is the empirical distribution function of the variable
\[X = \frac{\size(\MM_0)\,\vol_t(\TT,\,\VV_t-\VV_0)}{\area(\MM_0)},\]
called the \textbf{element-wise signed deviation}, measured from the set of all elements $\TT\in\MM_0$:
\[F_X(x) = \frac{1}{\size(\MM)}\, \card \left\{   \TT\in\MM_0 \,\left|\,
\frac{\size(\MM)\,\vol_t(\TT,\,\VV_t-\VV_0)}{\area(\MM)}\right. \leqslant x \right\},
\quad x\in\RR.\]
\index{Comparison between two meshes!Element-wise signed deviation}
\index{Comparison between two meshes!Deviation distribution function}

\begin{remark}
The empirical mean of the element-wise signed deviation is equal to the mean signed deviation defined in
\ref{def:MSD}:
\begin{align*}
\EE(X) & = \frac{1}{\size(\MM_0)}\,\sum_{\TT\in\MM_0}
 \left(\frac{\size(\MM_0)\,\vol_t(\TT,\,\VV_t-\VV_0)}{\area(\MM_0)}\right) \\
 & = \sum_{\TT\in\MM_0}\left(\frac{\vol_t(\TT,\,\VV_t-\VV_0)}{\area(\MM_0)}\right) \\
 & = \frac{\vol_t(\MM_0,\,\VV_t-\VV_0)}{\area(\MM_0)}\\
 & = \mathrm{MSD}_{\MM_0,\VV_0} (\MM_0,\,\VV_t),
\end{align*}
and is therefore invariant under sub-triangulation.
\end{remark}


% \bibliographystyle{plain}
% \bibliography{mybib.bib}
%

\end{document}
